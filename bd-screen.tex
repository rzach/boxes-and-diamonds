% bd-screen.tex

% driver file bd-screen.tex to produce the Boxes and Diamonds textbook on
% with same type block as in printed version, but but with on-screen
% features (color, links, etc).

% We use the memoir class for maximal flexibility of layout, but any
% class will do

\documentclass[11pt,openany]{memoir}

\RequirePackage{amsthm}
\RequirePackage{xcolor}
\RequirePackage{mdframed}
\RequirePackage[full]{textcomp}
\RequirePackage{gitinfo2}

\RequirePackage[pdftex,breaklinks,bookmarks,bookmarksopen,bookmarksopenlevel=1,colorlinks,urlcolor=dark-gray,linkcolor=reflex-blue]{hyperref}

%\usepackage[nonumberlist,toc,style=index]{glossaries}
%\makeglossaries


% let's set the whole thing in Baskervald X, with Universalis ADF
% Standard for sans-serif, and spread the lines a bit to make the text
% more readable

\usepackage[sfdefault]{universalis}
\usepackage[osf]{Baskervaldx} % oldstyle figures
\usepackage[bigdelims,baskervaldx]{newtxmath} 
\usepackage[cal=boondoxo]{mathalfa}

% Make sure we have a copyright symbol

\def\copyright{\textcircled{C}}

\def\oljobname{bd}

% next line needed for Adobe Acrobat to open the PDF.  may have
% somethign to do with transparency in the PNG graphics files?

\pdfminorversion=4

% load textpos for the cover

\usepackage[absolute,overlay]{textpos}

% set stock & paper size with narrow margins

\setstocksize{22cm}{17cm}

\settrimmedsize{\stockheight}{\stockwidth}{*}
\settrims{0pt}{0pt}

% let's calculate the line length for 65 characters in \normalfont

\setlxvchars

% set the size of the type block to golden ratio calculated width

\settypeblocksize{*}{1.05\lxvchars}{1.62}

% set spine and and edge margin

\setlrmargins{*}{*}{1}
\setulmargins{60pt}{*}{*}
\setheaderspaces{*}{*}{1}

\checkandfixthelayout

% Chapter style

\makeatletter

\definecolor{dkleadbeater}{RGB}{173, 94, 153}
\definecolor{leadbeater}{RGB}{124,67,110}
\definecolor{ltleadbeater}{RGB}{239,189,227}
\newlength{\barlength}
\makechapterstyle{leadbeater}{%
  \setlength{\afterchapskip}{40pt}
  \setlength{\beforechapskip}{50pt}
    \setlength{\midchapskip}{10pt}
  \renewcommand*{\afterchapternum}{\par\nobreak\vskip 0pt}
  \renewcommand*{\chapnamefont}{\fontsize{14pt}{0pt}\selectfont\sffamily\bfseries}
  \let\chapnumfont\chapnamefont
  \renewcommand*{\chaptitlefont}{\normalfont\fontsize{48pt}{48pt}\selectfont\bfseries\itshape\color{leadbeater}}
  \renewcommand*{\printchaptername}{%
    \chapnamefont\MakeUppercase{\@chapapp}}
  \renewcommand*{\printchaptertitle}[1]{%
    \chaptitlefont ##1\\[-\baselineskip]%
    \hspace*{-20pt}%
    \smash{\color{leadbeater}\rule{7pt}{300pt}}}
}

\renewcommand*{\partnamefont}{\fontsize{24pt}{0pt}\selectfont\bfseries\sffamily}
\renewcommand*{\partnumfont}{\fontsize{24pt}{0pt}\selectfont\bfseries\sffamily}
\renewcommand*{\parttitlefont}{\normalfont\fontsize{54pt}{54pt}\selectfont\bfseries\itshape\color{leadbeater}}
\renewcommand*{\printpartname}{%
  \partnamefont PART}

\makeatother

\chapterstyle{leadbeater}

\copypagestyle{leadbeater}{headings}

\makeoddhead{leadbeater}{\small\sffamily\color{leadbeater}\rightmark}{}
            {\color{leadbeater}\sffamily\bfseries\thepage}

\makeevenhead{leadbeater}{\small\sffamily\color{leadbeater}\leftmark}{}
            {\color{leadbeater}\sffamily\bfseries\thepage}

\usepackage[font={small,it}]{caption}


% \olpath has to point to the location of the OLP main
% directory/folder.  We're compiling from subdirectory
% courses/phil379/, so the main directory is two
% levels up.

\newcommand{\olpath}{../../}

% load all the Open Logic definitions. This will also load the
% local definitions in open-logic-sample-config.sty

\input{\olpath/sty/open-logic.sty}

% we want all the problems deferred to the very end

\input{\olpath/sty/open-logic-defer.sty}

% colors for links

\hypersetup{
colorlinks,
  allcolors=dkleadbeater,
  pdftitle={Boxes and Diamonds},
  pdfauthor = {Open Logic Project}
}

% load glossary entries

%\loadglsentries{include/glossary}

% end preamble

\input{includeonly}

\begin{document}

% !TeX root = ./bd-screen.tex

% Cover Page

\thispagestyle{empty}

\textblockcolor{ltleadbeater}
\begin{textblock*}{\stockwidth}(0in,3.5in)
  \vbox to 4.5in{\quad}
\end{textblock*}

\textblockcolor{}
% make front cover
\begin{textblock*}{\stockwidth}(0in,0in)
  \noindent\hfill
  \begin{minipage}[b][\stockheight][s]{.9\stockwidth}
    \begin{raggedleft}
      \vspace*{1.7cm}
      \hfill
      \sffamily\fontsize{50pt}{50pt}\selectfont
      \color{dkleadbeater}
      \textbf{Boxes and Diamonds}%
                    
      \vspace*{1cm}
      \color{black}
      \sffamily
      \fontsize{25pt}{25pt}\selectfont
      \textbf{An Open Introduction to\\ Modal Logic}
      \vskip2cm

      \hfill\includegraphics[width=.29\textwidth]{\olpath/assets/portraits/barcan-circle.pdf}
      \includegraphics[width=.29\textwidth]{\olpath/assets/portraits/carnap-circle.pdf}\hfill{}
      
      \includegraphics[width=.29\textwidth]{\olpath/assets/portraits/antonelli-circle.pdf}\hfill
      \includegraphics[width=.29\textwidth]{\olpath/assets/portraits/lewis-circle.pdf}\hfill
      \includegraphics[width=.29\textwidth]{\olpath/assets/portraits/prior-circle.pdf}
      
      \hfill\includegraphics[width=.29\textwidth]{\olpath/assets/portraits/heyting-circle.pdf}
      \includegraphics[width=.29\textwidth]{\olpath/assets/portraits/kripke-circle.pdf}\hfill{}
      
      \vspace{-.5cm}
      \fontsize{16pt}{19pt}\selectfont
      \textbf{Fall 2019$\boldsymbol{\alpha}$}\par
      \vspace{1cm}
    \end{raggedleft}
  \end{minipage}
  \hfill{}
\end{textblock*}
\ 
\newpage
\setcounter{page}{1}
\nopagecolor




% Now load the actual text

% !TeX root = ./bd-screen.tex
% bd.tex
%
% driver file bd.tex to produce text

\preto\OLEndChapterHook{\IfFileExists{include/summary-\thechapter}
        {\section*{Summary}\addcontentsline{toc}{section}{Summary}
        \let\emph\textbf\input{include/summary-\thechapter}\let\emph\textit}{}}

\problemsperchapter

\allowdisplaybreaks

\frontmatter

\OLPfrontmatter

% !TeX root = ../bd-screen.tex

\chapter{Preface}

This is an introductory textbook on modal logic. I use it as the main
text when I teach Philosophy~579.2 (Modal Logic) at the University of
Calgary. It is based on material from the
\href{https://openlogicproject.org}{Open Logic Project}.

The main text assumes familiarity with some elementary set theory and
the basics of (propositional) logic. This material is part of a
prerequisite for my course, Logic~II. The textbook for that course,
\href{https://slc.openlogicproject.org}{\emph{Sets, Logic,
Computation}}, is also based on the OLP, and so is available for free.
The required material is included as appendices in this book,
however. I assign these appendices for background reading whenever I
teach the material.

\Cref{nml:::part} is originally based in part on Aldo Antonelli's
lecture notes on ``Classical Correspondence Theory for Basic Modal
Logic,'' which he contributed to the OLP before his untimely death in
2015. I heavily revised and expanded these notes, e.g., the material
on frame definability and tableaux is new.

% !TeX root = ../bd-screen.tex

\chapter{Introduction}

Modal logics are extensions of classical logic by the operators $\Box$
(``box'') and $\Diamond$ (``diamond''), which attach to !!{formula}s.
Intuitively, $\Box$ may be read as ``necessarily'' and $\Diamond$ as
``possibly,'' so $\Box p$ is ``$p$ is necessarily true'' and $\Diamond
p$ is ``$p$ is possibly true.'' As necessity and possibility are
fundamental metaphysical notions, modal logic is obviously of great
philosophical interest. It allows the formalization of metaphysical
principles such as ``$\Box p \lif p$'' (if $p$ is necessary, it is
true) or ``$\Diamond p \lif \Box\Diamond p$'' (if $p$ is possible,
it is necessarily possible).

The operators $\Box$ and $\Diamond$ are \emph{intensional}. This means
that whether $\Box !A$ or $\Diamond !A$ holds does not just depend on
whether $!A$ holds or doesn't.  An operator which is not intensional
is \emph{extensional}. Negation is extensional: $\lnot !A$ holds iff
$!A$ does not; so whether $\lnot !A$ holds only depends on whether
$!A$ holds or doesn't. $\Box$ and $\Diamond$ are not like that:
whether $\Box !A$ or $\Diamond !A$ holds depends also on the meaning
of $!A$.  While ordinary truth-functional semantics is enough to deal
with extensional operators, intensional operators like $\Box$ and
$\Diamond$ require a different kind of semantics. One such semantics
which takes center stage in this book is relational semantics (also
called possible-worlds semantics or Kripke semantics). 

For the logic which corresponds to the interpretation of $\Box$ as
``necessarily,'' this semantics is relatively simple: instead of
assigning truth values to !!{propositional variable}s, an
interpretation~$\mModel{M}$ assigns a set of ``worlds'' to
them---intuitively, those worlds~$w$ at which $p$ is interpreted as true.
On the basis of such an interpretation, we can define a satisfaction
relation. The definition of this satisfaction relation for
!!{formula}s $\Box !A$ make $!A$ satisfied at a world~$w$ iff $!A$ is
satisfied at \emph{all} worlds: $\mSat{M}{\Box !A}[w]$ iff
$\mSat{M}{!A}[v]$ for all worlds~$v$. This corresponds to Leibniz's
idea that what's necessarily true is what's true in every possible world.

``Necessarily'' is not the only way to interpret the $\Box$ operator,
but it is the standard one---``necessarily'' and ``possibly'' are the
so-called \emph{alethic} modalities. Other interpretations read $\Box$
as ``it is known (by some person~$A$) that,'' as ``some person $A$
believes that,'' ``it ought to be the case that,'' or ``it will always
be true that.'' These are epistemic, doxastic, deontic, and temporal
modalities, respectively. Different interpretations of $\Box$ will
make different !!{formula}s as logically true, and different
inferences as valid. For instance, everything necessary and everything
known is true, so $\Box !A \lif !A$ is a truth on the alethic and
epistemic interpretations. By contrast, not everything believed nor
everything that ought to be the case actually is the case, so $\Box !A
\lif !A$ is not a truth on the doxastic or deontic interpretations. 

In order to deal with different interpretations of the modal
operators, the semantics is extended by a relation between worlds, the
so-called accessibility relation.  Then $\mSat{M}{!A}[w]$ if
$\mSat{M}{!A}[v]$ for all worlds $v$ which are accessible from~$w$.
The resulting semantics is very versatile and powerful, and the basic
idea can be used to provide semantic interpretations for logics based
on other intensional operators. One such logic is intuitionistic
logic, a constructive logic based on L. E. J. Brouwer's branch of
constructive mathematics. Intuitionistic logic is philosophically
interesting for this reason---it plays an important role in
constructive accounts of mathematics---but was also propsed as a logic
superior to classical logic by the influential English philosopher
Michael Dummett in the 20th century. Another application of relational
models is as a semantics for subjunctive, or counterfactual,
conditionals, an approach pioneered by Robert Stalnaker and David K.
Lewis.

This book is an introduction to the syntax, semantics, and proof
theory of intensional logics. It only deals with propositional logics,
although future editions will also treat predicate logics.  The
material is divided into three parts: The first part deals with normal
modal logics. These are logics with the operators $\Box$ and
$\Diamond$. We discuss their syntax, relational models and semantic
notions based on them (such as validity and consequence) and proof
systems (both axiomatic systems and tableaux). We establish some
basic results about these logics, such as the soundness and
completeness of the proof systems considered, and discuss some
model-theoretic constructions such as filtrations. The second part
deals with intuitionistic logic. Here we discuss natural deduction and
axiomatic derivations, relational and topological semantics, and
soundness and completeness of the proof systems. The third part deals
with the Lewis-Stalnaker semantics of counterfactual conditionals. 
The appendix discusses some ideas and results from set theory
and the theory of relations that's crucial to the relational
semantics, as well as reviews syntax, semantics, and proof theory of
classical propositional logic.

\mainmatter

\olimport*[normal-modal-logic]{normal-modal-logic}

\part{Intuitionistic Logic}

\olimport*[intuitionistic-logic/introduction]{introduction}

\olimport*[intuitionistic-logic/semantics]{semantics}

\olimport*[intuitionistic-logic/soundness-completeness]{soundness-completeness}

\olimport*[intuitionistic-logic/tableaux]{tableaux}

\olimport*[counterfactuals]{counterfactuals}

\olimport*[applied-modal-logic]{applied-modal-logic}

\part{Appendices}

\appendix

\olimport*[sets-functions-relations/sets]{sets}
\olimport*[sets-functions-relations/relations]{relations}

\tagfalse{FOL}
\olimport*[propositional-logic/syntax-and-semantics]{syntax-and-semantics}

\chapter{Axiomatic \usetoken{P}{derivation}}

\olimport*[first-order-logic/proof-systems]{introduction}
\olimport*[first-order-logic/proof-systems]{axiomatic-deduction}
\olimport*[first-order-logic/axiomatic-deduction]{rules-and-proofs}
\olimport*[first-order-logic/axiomatic-deduction]{axioms-rules-propositional}
\olimport*[first-order-logic/axiomatic-deduction]{proving-things}
\olimport*[first-order-logic/axiomatic-deduction]{proof-theoretic-notions}
\olimport*[first-order-logic/axiomatic-deduction]{deduction-theorem}
\olimport*[first-order-logic/axiomatic-deduction]{provability-consistency}
\olimport*[first-order-logic/axiomatic-deduction]{provability-propositional}
\olimport*[first-order-logic/axiomatic-deduction]{soundness}

\OLEndChapterHook

\chapter{Tableaux}

\olimport*[first-order-logic/proof-systems]{tableaux}
\olimport*[first-order-logic/tableaux]{rules-and-proofs}
\olimport*[first-order-logic/tableaux]{propositional-rules}
\olimport*[first-order-logic/tableaux]{derivations}
\olimport*[first-order-logic/tableaux]{proving-things}
\olimport*[first-order-logic/tableaux]{proof-theoretic-notions}
\olimport*[first-order-logic/tableaux]{provability-consistency}
\olimport*[first-order-logic/tableaux]{provability-propositional}
\olimport*[first-order-logic/tableaux]{soundness}

\OLEndChapterHook

\chapter{The Completeness Theorem}

\olimport*[first-order-logic/completeness]{introduction}
\olimport*[first-order-logic/completeness]{outline}
\olimport*[first-order-logic/completeness]{complete-consistent-sets}
\olimport*[first-order-logic/completeness]{lindenbaums-lemma}
\olimport*[first-order-logic/completeness]{construction-of-model}
\olimport*[first-order-logic/completeness]{completeness-thm}

\OLEndChapterHook

\stopproblems
\def\ifproblems#1{}

\backmatter

\clearpage
%\photocredits

\bibliographystyle{\olpath/bib/natbib-oup}

%\bibliography{\olpath/bib/open-logic.bib}

\olimport*{\olpath/content/open-logic-about}

\end{document}

